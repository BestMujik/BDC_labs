\section*{DATABASE NORMALIZATION}
\phantomsection

All the objects and implimentation of each practical exercise must be included in a T-SQL script.
The final user should be able to run the script multiple times without errors. 

Note: The exported script will not be considered.

\section{Design of the database}

Design the database that will be in the 3NF or BCNF. Attach the diagram of this database. It should include 3 types of relationships:


\begin{enumerate}[itemsep=2pt]
\item 1:1;
\item 1:n;
\item n:n;
\end{enumerate}

Describe all the entities from business point of view and the role of each defined relationship.

\section{Table Creation and Modification}

The database must have:
\begin{enumerate}[itemsep=2pt]
\item 4 different database constraints;
\item 1 default value;
\item minimum 5 records per table;
\item the tabled should be included in 2 different schemes;
\item for 3 key elements of the database to be defined synonyms;
\end{enumerate}

\section*{MANIPULATION OF THE DATABASE}
\phantomsection

\section{SELECT query in T-SQL}\label{sec:select}

\begin{enumerate}[itemsep=2pt]
\item Write a query on you database that will use AGGREGATION FUNCTION, GROUP BY and HAVING clause;
\item Show an example of a query that use subqueries in other places except WHERE ;
\end{enumerate}

Attach the screens with the output of the queries.

\section{Stored procedures and function defined by user}

1. Create a user defined function based on example A from previous exercise

2. Create a stored procedure based on example B from previous exercise. 

\section{Indexes}

Increse the performance of your database by creating an idex for one of the execrcises from previous section \hyperref[sec:select]{SELECT query in T-SQL}. 

Show the results of this optimization using the performance excution plan before and after the index have been created (attach screens if it is necessary).


\section{View administration}

Create a view bazed on one of your scheme. Use one of the options: CHECK OPTION or SCHEMA BINDING. 

Show with examples how it is working (attach screens if it is necessary).

\section{Triggers}

Create a trigger that will forbid modification of the column that have DEFAULT value.

\section{Database backup and restore}

Create a differenatial backup of your database. Restore it to a new database named restoredDB.

\section{Script}

Included the script that will realize all the tasks.

\lstinputlisting[style=mystyle, language=SQL, caption={Database NAMEOFYOURDATABASE}, label=list1]{sourcecode/script.sql}

\clearpage